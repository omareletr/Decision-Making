\documentclass[11pt]{article}
\usepackage[pdftex]{graphicx}
% \usepackage[margin=1in]{geometry}
\usepackage{MnSymbol}
\usepackage{setspace}
\singlespacing
\begin{document}

\section{Designing Brain-Computer Interfaces for Intelligent Information
Delivery
Systems}\label{designing-brain-computer-interfaces-for-intelligent-information-delivery-systems}

\subsection{Peck - Thesis}\label{peck---thesis}

\subsection{Chapter One -
Introduction}\label{chapter-one---introduction}

\begin{itemize}
\item
  Def: brain-computer interface (BCI) (via thought process) (not
  keyboard, mouse, etc.)

  \begin{itemize}
  \itemsep1pt\parskip0pt\parsep0pt
  \item
    Alternative input
  \item
    Direct-control BCI: brain as primary input

    \begin{itemize}
    \itemsep1pt\parskip0pt\parsep0pt
    \item
      Application: people without full motor cap.
    \end{itemize}
  \end{itemize}
\item
  Into the consumer space (working environment, everyday user, etc.)

  \begin{itemize}
  \itemsep1pt\parskip0pt\parsep0pt
  \item
    Direct: still not as efficient as keyboard and mouse for ordinary
    folks
  \item
    Alt to augmentation (a recent shift) - \emph{passive brain-computer
    interfaces} (pBCI) has advantages

    \begin{itemize}
    \itemsep1pt\parskip0pt\parsep0pt
    \item
      Nice example
    \item
      doesn't interfere with normal behavior
    \item
      Augmentative input beneficial for information delivery: subtly of
      when and what

      \begin{itemize}
      \itemsep1pt\parskip0pt\parsep0pt
      \item
        Analogy: social interaction / treating computers as humans
      \end{itemize}
    \end{itemize}
  \item
    Difficulty in applying pBCI:

    \begin{itemize}
    \itemsep1pt\parskip0pt\parsep0pt
    \item
      Very little work done
    \item
      Current rain sensing technologies are cumbersome, constraining
    \item
      Monitoring physiological data problematic to interface
      designers????
    \item
      Brain data noisy thus difficult to interpret

      \begin{itemize}
      \itemsep1pt\parskip0pt\parsep0pt
      \item
        Automatic detection of user's state non-trivial
      \end{itemize}
    \end{itemize}
  \end{itemize}
\end{itemize}

( $\lefthalfcup$ Question: necessity? )

\subsubsection{1.2 BCI for info delivery}\label{bci-for-info-delivery}

\begin{itemize}
\itemsep1pt\parskip0pt\parsep0pt
\item
  Manifested problem : technology is sometimes distractive and not
  living up to people's expectations. (Example, stats)

  \begin{itemize}
  \itemsep1pt\parskip0pt\parsep0pt
  \item
    Why being distracted bad (info delivery specific)
  \item
    People asking more info; easier access, too
  \end{itemize}
\item
  Problem ID : As info increase, how can tech prevent ``overloading''?
\item
  BUT, brain has mechanism to handle info effectively in social context
  by detecting clues?
\item
  Devices should understand social rules so as to better serve people.
\item
  Problem ID: Computers don't understand us enough, (compared to social
  interactions)

  \begin{itemize}
  \itemsep1pt\parskip0pt\parsep0pt
  \item
    Source: insufficient input (e.g.~keyboard)
  \item
    efficient delivery of info --\textgreater{} Only Understanding
  \end{itemize}
\item
  Solution: new ways to communicating to the computer (e.g.)
\item
  Practical level: a certain method

  \begin{itemize}
  \itemsep1pt\parskip0pt\parsep0pt
  \item
    Challenges of this method, in terms of previous research,
    feasibility, outcome?
  \end{itemize}
\end{itemize}

\subsubsection{1.3 Outline}\label{outline}

\begin{itemize}
\itemsep1pt\parskip0pt\parsep0pt
\item
  Brain sensing can be used to capture \emph{how} info is presented
\item
  System to decide \emph{which} info is presented
\item
  System += when to deliver
\item
  Strategies for processing brain data
\end{itemize}

Name of the game: physiological data perhaps will support intelligent
info delivery system

($\lefthalfcup$ Question: what is an interface? A more solid description) (Brains? What
is the subject?) (\emph{neutral} predictions, etc.?)

\section{Heart and mind in conflict: the interplay of affect and
cognition in consumer decision
making}\label{heart-and-mind-in-conflict-the-interplay-of-affect-and-cognition-in-consumer-decision-making}

\subsection{Abstract}\label{abstract}

Decision making is influenced by

\begin{itemize}
\itemsep1pt\parskip0pt\parsep0pt
\item
  Automatically evoked task-induced affect
\item
  Cognitions that are generated in a more controlled manner on exposure
  to alternatives in a choice task
\end{itemize}

The experiment:

Chocolate cake (intense positive affect, less favourable result
cognition)


\begin{itemize}
\itemsep1pt\parskip0pt\parsep0pt
  \item 

\end{itemize}

($\lefthalfcup$ Q: cognitive control)

\emph{def}:Executive functions (also known as cognitive control and
supervisory attentional system) is an umbrella term for the management
(regulation, control) of cognitive processes,{[}1{]} including working
memory, reasoning, task flexibility, and problem solving {[}2{]} as well
as planning and execution.{[}3{]}

The executive system is a theorized cognitive system in psychology that
controls and manages other cognitive processes, such as executive
functions. The prefrontal areas of the frontal lobe are necessary but
not solely sufficient for carrying out these functions.{[}4{]}


\section{air-traffic controller}
http://www.skybrary.aero/bookshelf/books/1643.pdf

\end{document}